\documentclass[12pt]{article}
\usepackage[utf8]{inputenc}
\usepackage[brazil]{babel}
\usepackage{geometry}
\geometry{a4paper, margin=2.5cm}
\usepackage{setspace}
\usepackage{indentfirst}
\usepackage{graphicx}
\usepackage{url}

% Espaçamento entre linhas
\onehalfspacing

\begin{document}

% ================== CAPA ==================
\begin{titlepage}
    \centering
    % Logo da PUC Minas (faça upload do arquivo "logopucminas.png")
    \includegraphics[width=0.25\textwidth]{logopucminas.jpg}\par\vspace{1cm}
    
    {\bfseries\Large PONTIFÍCIA UNIVERSIDADE CATÓLICA DE MINAS GERAIS\par}
    \vspace{0.3cm}
    {\large Instituto de Ciências Exatas e de Informática\par}
    
    \vfill
    {\bfseries\LARGE Relatório de Grafos\par}
    \vspace{2cm}
    
    {\normalsize
    Alessandra Faria Rodrigues\textsuperscript{1}\\
    Enzo Marques Pylo\textsuperscript{2}\\
    Débora Luiza de Paula Silva\textsuperscript{3}\\
    Douglas Nicolas Silva Gomes\textsuperscript{4}\\
    Gabriel Chaves Mendes\textsuperscript{5}\\
    Guilherme Henrique da Silva Teodoro\textsuperscript{6}\\
    Maria Eduarda P. Martins\textsuperscript{7}\\
    Suzane Lemos de Lima\textsuperscript{8}\\
    }
    
    \vfill
    Belo Horizonte, 30 de agosto de 2025
\end{titlepage}

% ================== RESUMO ==================
\begin{abstract}
Este relatório apresenta o desenvolvimento e a implementação de estruturas para a manipulação 
de grafos em linguagem C/C++. A implementação abrange quatro tipos distintos de grafos: (i) 
não direcionado não ponderado; (ii) não direcionado ponderado; (iii) direcionado não ponderado; 
e (iv) direcionado ponderado. São discutidos os detalhes da implementação, com ênfase na 
escolha da lista de adjacência como estrutura de dados para a representação dos grafos, e as 
funcionalidades básicas desenvolvidas para a manipulação de vértices e arestas em cada uma das 
abordagens. 

\textbf{Palavras-chave:} Grafos. Lista de Adjacência. Grafos Ponderados. Grafos Direcionados.
\end{abstract}

% ================== INTRODUÇÃO ==================
\section{Introdução}
A representação de grafos é um conceito fundamental em Ciência da Computação, 
permitindo a modelagem de sistemas complexos e a relação entre diversas áreas. Um grafo é 
definido como um conjunto de vértices e arestas, cujas características, como direção e peso, 
determinam o tipo e a finalidade da estrutura. A implementação dessas diferentes variações de 
grafos requer a escolha de uma representação de dados eficiente, como a lista de adjacência, que 
impacta diretamente o desempenho e a funcionalidade para a resolução de problemas.  

A principal diferença entre os tipos de grafos está na natureza das suas ligações. Um grafo 
não direcionado é um conjunto de vértices e arestas, onde as arestas não têm direção e conectam 
dois vértices de forma simétrica, permitindo a ligação entre os nós em ambos os sentidos. Por 
outro lado, em um grafo direcionado as ligações têm um sentido único, onde uma ligação de A 
para B não garante uma ligação de volta de B para A, semelhante a uma rua de sentido único. 

Outra classificação dos grafos baseia-se na presença ou ausência de valores nas suas 
arestas. Em um grafo não ponderado, as arestas não possuem um valor numérico ou peso associado 
a elas. Já um grafo ponderado atribui um ``peso'' ou ``custo'' a cada aresta, que pode representar 
distância, tempo ou qualquer outra métrica. Esta característica é crucial para problemas de 
otimização, como encontrar o caminho mais curto entre dois pontos num mapa. 

Neste trabalho, diversas responsabilidades foram distribuidas aos membros do grupo. Os integrantes Enzo Pylo, Débora Luiza, Douglas Nicolas e Guilherme Teodoro ficaram responsaveis pelo desenvolvimento do código e implementação dos grafos, os integrantes Gabriel Chaves e Maria Eduarda pelo desenvolvimento do relatório e as integrantes Alessadra Faria e Suzane Lemos pela revisão e alteração do relatório.

% ================== METODOLOGIA ==================
\section{Metodologia e Implementação}

Neste trabalho foi utilizada apenas uma única abordagem.

\subsection{Abordagem por Lista de Adjacência}
Esta abordagem de representação consiste em associar cada vértice do grafo a uma lista contendo 
todos os seus vizinhos, ou seja, aqueles diretamente conectados por uma aresta. Apesar de exigir 
uma busca sequencial para verificar a adjacência entre dois vértices específicos, sua principal 
vantagem está na eficiência de memória, especialmente para grafos esparsos (aqueles que têm o 
número de arestas proporcional ao número de vértices). Na implementação desta abordagem, são 
realizadas as seguintes operações:

\begin{itemize}
    \item Utilização de um vetor ou arranjo principal, onde cada índice corresponde a um vértice do grafo.
    \item Para cada vértice $v$, o armazenamento de uma lista (geralmente uma lista encadeada) contendo os identificadores de todos os vértices $u$ para cada aresta existente $(v, u)$.
    \item Em grafos ponderados, a lista armazena pares (vértice, peso), representando não apenas o vizinho, mas também o peso da aresta que os conecta.
\end{itemize}

\subsection{Implementação}

\subsubsection{Grafo Direcionado Não Ponderado}
Em um vetor, o índice representa o vértice de origem e cada índice contém uma 
lista armazenando todos os vértices de destino conectados por uma 
aresta. Ao chamar a função \texttt{adicionarAresta(v1, v2)}, ela cria uma aresta que conecta 
v1 até v2 e adiciona v2 à lista de v1, por meio da função \texttt{adj[v1].push\_back(v2)}, 
existindo assim um caminho de v1 para v2, mas não ao contrário.

\subsubsection{Grafo Direcionado Ponderado}
Assim como o anterior, em um vetor o índice representa um vértice de origem e 
cada índice contém uma lista armazenando todos os vértices de destino. 
Entretanto, ao adicionar uma nova aresta, \texttt{adicionarAresta(v1, v2, peso)}, um novo 
parâmetro (peso) é passado para a lista de adjacência de v1, armazenando um par 
contendo o vértice de destino e o peso da aresta: \texttt{adjPeso[v1].emplace\_back(v2, peso)}.

\subsubsection{Grafo Não Direcionado Não Ponderado}
Para a implementação desse grafo, quando é chamada a função 
\texttt{adicionarAresta(v1, v2)}, são adicionadas duas arestas direcionadas: uma de v1 para 
v2 e outra de v2 para v1, por meio das funções \texttt{adj[v1].push\_back(v2)} e 
\texttt{adj[v2].push\_back(v1)}. Assim, simula-se um grafo não direcionado.

\subsubsection{Grafo Não Direcionado Ponderado}
Assim como o grafo não direcionado não ponderado, são adicionadas duas arestas 
direcionadas, uma de v1 para v2 e outra de v2 para v1. Entretanto, é passado 
um parâmetro a mais (peso), adicionando o par $(v2, peso)$ na lista de v1 e o par $(v1, peso)$ 
na lista de v2.  

% ================== CONCLUSÃO ==================
\section{Conclusão}
Este trabalho concluiu a implementação de quatro tipos de grafos, demostrando que a 
representação por lista de adjacência é uma abordagem eficaz. A estratégia de reutilizar a estrutura 
do grafo direcionado para construir o não direcionado mostrou-se uma solução prática e de fácil 
desenvolvimento para a construção dos grafos.

% ================== REFERÊNCIAS ==================
\section*{Referências}

\begin{itemize}
    \item Böther, M., et al. Efficiently Computing Directed Minimum Spanning Trees. In 2023 Proceedings of the Symposium on Algorithm Engineering and Experiments (ALENEX), pp. 86--95, 2023.
    \item Edmonds, J. Optimum Branchings, Journal of Research of the National Bureau of Standards Section B, 71B (4): 233--240, 1967.
    \item Gabow, H. N.; Galil, Z.; Spencer, T.; Tarjan, R. E. Efficient algorithms for finding minimum spanning trees in undirected and directed graphs, Combinatorica, 6 (2): 109--122, 1986.
    \item Tarjan, R. E. Finding Optimum Branchings, Networks, 7: 25--35, 1977.
    \item https://pt.stackoverflow.com/questions/216105/todos-poss\%C3\%ADveis-caminhos-em-grafos
\end{itemize}


\end{document}
