\documentclass[12pt]{article}
\usepackage[utf8]{inputenc}
\usepackage[brazil]{babel}
\usepackage{geometry}
\geometry{a4paper, margin=2.5cm}
\usepackage{setspace}
\usepackage{indentfirst}
\usepackage{amsmath}
\usepackage{amssymb}

\onehalfspacing

\begin{document}

% ================================================================================
% RELATÓRIO DE IMPLEMENTAÇÃO
% ================================================================================

\section{Implementação}

\subsection{Estruturas de Dados}

A representação dos grafos utiliza \textbf{listas de adjacência}, adequada para grafos esparsos. A estrutura \textit{Union-Find} com \textit{path compression} e \textit{union by rank} é empregada tanto no pré-processamento (agrupamento de pixels) quanto nos algoritmos para detecção e contração de ciclos, garantindo complexidade amortizada $O(\alpha(n))$ por operação.

\subsection{Estratégia de Superpixels}

A modelagem direta de pixels como vértices ($V = N \times M$) resultaria em complexidade proibitiva. Adotamos uma estratégia de pré-processamento baseada em \textbf{Superpixels}: pixels vizinhos com similaridade de cor (distância Euclidiana RGB $< 15$) são agrupados via \textit{Union-Find}, formando vértices lógicos. Esta técnica reduziu grafos de $10^5$ pixels para $10^2$--$10^3$ supernós (redução de 99\%), viabilizando execução em tempo real. Um filtro de média (\textit{Box Blur}) é aplicado previamente para reduzir ruído e evitar super-segmentação.

\subsection{Algoritmo de Edmonds}

O algoritmo clássico de Edmonds (Chu-Liu, 1965) foi implementado como referência, seguindo a formulação recursiva com seleção gulosa, detecção de ciclos e contração de super-nós. Sua complexidade $O(VE)$ o torna inadequado para grafos maiores, motivando a implementação dos métodos otimizados.

\subsection{Algoritmo de Tarjan (1977)}

O algoritmo de Tarjan melhora a complexidade através do uso de \textbf{filas de prioridade mescláveis}. A ideia central é manter, para cada componente do grafo, uma heap contendo todas as arestas de entrada ordenadas por peso.

\textbf{Fase de Contração:} O algoritmo processa vértices iterativamente. Para cada vértice $v$, seleciona-se a aresta de entrada de menor peso da heap correspondente. Se esta aresta forma um ciclo (detectado via marcação de estados), os componentes do ciclo são contraídos em um super-nó. As heaps dos componentes são \textit{fundidas} (operação \textit{meld}), e os pesos são ajustados via \textbf{propagação preguiçosa} (\textit{lazy propagation}): ao invés de atualizar cada aresta individualmente, armazena-se um valor de ajuste no nó raiz da heap, propagado apenas quando necessário.

\textbf{Remoção de Auto-loops:} Durante a seleção da aresta mínima, arestas que se tornaram auto-loops (origem e destino no mesmo componente após contrações) são descartadas da heap.

A complexidade resultante é $O(E \log V)$, onde o fator logarítmico deriva das operações de heap.

\subsection{Algoritmo de Gabow, Galil, Spencer e Tarjan (1986)}

O algoritmo de Gabow et al. refina a abordagem de Tarjan através da técnica de \textbf{Path Growing} (Crescimento de Caminho) e uma estrutura hierárquica para reconstrução da solução.

\textbf{Path Growing:} Ao invés de processar vértices isoladamente, o algoritmo estende ``caminhos'' a partir de vértices não processados. Cada vértice assume um de três estados: \textit{novo}, \textit{ativo} (em processamento) ou \textit{processado}. Quando um caminho encontra um vértice ativo, um ciclo é detectado e contraído.

\textbf{Contração Hierárquica:} Ciclos são registrados em uma pilha com informações sobre seus componentes e as arestas que os conectam. Cada super-nó mantém referência ao seu ``pai'' na hierarquia de contrações, permitindo rastrear a qual ciclo original cada vértice pertence.

\textbf{Expansão:} Após processar todo o grafo, a pilha de ciclos é desempilhada em ordem reversa. Para cada ciclo, determina-se qual sub-componente recebe a aresta externa (que entra no super-nó) e quais mantêm suas arestas internas originais. Esta fase reconstrói a arborescência no grafo original.

\subsection{Decisão de Projeto: Skew Heaps}

Para ambos os algoritmos (Tarjan e Gabow), optou-se por \textbf{Skew Heaps} em substituição aos Fibonacci Heaps sugeridos na literatura original. Esta decisão fundamenta-se nos princípios de Engenharia de Algoritmos (Böther et al., 2023): embora Fibonacci Heaps ofereçam complexidade teórica ótima $O(m + n \log n)$, suas constantes ocultas e \textit{overhead} de memória frequentemente resultam em desempenho inferior para instâncias práticas.

As Skew Heaps são árvores binárias auto-ajustáveis que suportam:
\begin{itemize}
    \item \texttt{merge(h1, h2)}: Fusão de duas heaps em $O(\log n)$ amortizado
    \item \texttt{findMin(h)}: Acesso ao mínimo em $O(1)$
    \item \texttt{deleteMin(h)}: Remoção do mínimo em $O(\log n)$ amortizado
\end{itemize}

A operação de fusão, essencial para contração de ciclos, é implementada recursivamente: compara-se as raízes, e a menor torna-se raiz da heap resultante, com seus filhos trocados (\textit{skew property}) para manter o balanceamento amortizado.

A implementação inclui \textbf{lazy propagation} para ajuste de pesos: quando um ciclo é contraído, ao invés de percorrer todas as arestas subtraindo o peso da aresta removida, armazena-se este valor no campo \texttt{lazy} da raiz, propagando-o aos filhos apenas durante acessos subsequentes.

Considerando que nossa estratégia de Superpixels já reduz o grafo para centenas de vértices, a complexidade $O(m \log n)$ das Skew Heaps é virtualmente indistinguível da ótima, garantindo código mais robusto e livre de vazamentos de memória.

\subsection{Algoritmo de Kruskal (Baseline)}

O algoritmo de Kruskal para MST não-direcionada foi implementado como \textit{baseline} comparativo, utilizando ordenação de arestas e \textit{Union-Find}. Os experimentos demonstraram que métodos direcionados (Tarjan, Gabow) preservam melhor gradientes de cor em transições suaves, validando a escolha da modelagem por arborescência.

\end{document}
